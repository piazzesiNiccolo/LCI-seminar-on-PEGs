\section{Parsing Expression Grammars: an introduction}

\begin{frame}
	\frametitle{Concrete definition}
	\textbf{Introduced by Bryan Ford in 2004 \cite{peg}.}
	
   \textbf{An alternative, recognition based, formal foundation for language syntax.}
   
   \textbf{Similar approach to EBNF notation, where a CFG is enriched with RE-like  features.}
   
\end{frame}

\begin{frame}
	\frametitle{A key difference}
	\begin{minipage}{\textwidth}
		The nondeterministic choice operator  '$|$' is substituted by a \emph{prioritized} choice operator '/'. 
	\end{minipage}
\vfill
%&
	\begin{minipage}{\textwidth}
	\begin{columns}
		\column{0.5\textwidth}
		\centering
		A  $\rightarrow$ ab $|$ a 
		
		$\Updownarrow$
		
		A $\rightarrow$ a $|$ ab
		\column{0.5\textwidth}
		\centering
		A  $\leftarrow$ ab $/$ a 
		
		$\not\Updownarrow$
		
		A $\leftarrow$ a $/$ ab
		
	\end{columns}
			

	\end{minipage}
\end{frame}

\begin{frame}
	\frametitle{Operators}
	\begin{columns}
	\column{0.5\textwidth}
	A parsing expression grammar consists of a set of definition of the form 'A $\leftarrow e $', where A is a nonterminal and $e$ is a \emph{parsing expression}.
	
	 A parsing expression can be constructed using the operators defined in Table \ref{tab:op}.
	\column{0.5\textwidth}
	\begin{table}
		\footnotesize
		\resizebox{\columnwidth}{!}{%
		\begin{tabular}{|c c c c|}
			\hline
			Operator &  Type & Precedence & Description\\
			\hline
			'        '	& Primary & 5 & Literal String \\ 
			"     "		& Primary & 5 & Literal String \\ 
			$\lbrack$  $\rbrack$			& Primary & 5 & Character class \\
			. 			& Primary  & 5 & Any character \\
			(e) 	& Primary & 5 & Grouping \\
			e? 		& Unary suffix & 4 & Optional \\
			e* 		& Unary suffix & 4 & Zero or more \\
			e+		& Unary suffix & 4 & One or more \\
			\&e  	& Unary prefix & 3 & And-predicate\\
			!e 			& Unary prefix & 3 & Not-predicate \\
			e1 e2	& Binary	& 2	&  Sequence \\
			e1 $/$ e2	& Binary	& 1	&  Prioritized choice\\
			\hline
	\end{tabular}}
	\caption{Operators for constructing parsing expression }
	\label{tab:op}
	\end{table}
\end{columns}
\end{frame}
\begin{frame}
	'   ' and " " delimit string literals, while [ ]
 indicate characters classes, which can also be specified using ranges  such as 'a-z' . The ' . ' constant matches any character.


?, * and + work as in normal regular expression, but they are greedy instead of non deterministic.


\& and ! are \emph{syntactic predicates\footnote{See \cite{parr1994adding}}.}  The expression '\&$e$ tries to match the given pattern and then unconditionally backtracks to the starting point, mantaining the knowledge of wether \emph{e} succeded or failed. '!\emph{e}' succeed if \emph{e} fails and viceversa. These two operators are fundamental for the expressive power of Parsing Expression Grammars, allowing them to describe languages that are not even parsed using CFGs (e.g $a^nb^nc^n$).
\end{frame}
\begin{frame}
	\frametitle{Peg describing its own ASCII syntax}
	\begin{minipage}{\textwidth}
		\begin{flushleft}Hierarchical syntax\end{flushleft}
\begin{columns}
		
		\column{0.5\textwidth}
		\tiny
	
		
		Grammar $\leftarrow$ Space Def+ EOF
		
		Def $\leftarrow$ LArrow Expr
		
		Expr  $\leftarrow$ Seq (Slash Seq)*
		
		Seq $\leftarrow$ Prefix*
		\column{0.5\textwidth}
		\tiny
		Prefix $\leftarrow$ (And / Not )? Suffix
		
		Suffix $\leftarrow$ Primary (Question / Star / Plus)?
		
		Primary $\leftarrow$ Identifier !LArrow  
												
												\textbf{/} Open  Expr Close
												
												\textbf{/}  Literal
												
												\textbf{/} Class
												 \textbf{/} Dot
		\end{columns}
		\end{minipage}
		\vfill
		%&
		\begin{minipage}{\textwidth}
			\begin{flushleft}Lexical syntax\end{flushleft}
			\tiny
		\begin{columns}
		\column{0.5\textwidth}
	
		Identifier $\leftarrow$ Istart Icont* Space
		
		Istart $\leftarrow$ [a-zA-Z\_]
		
		Icont $\leftarrow$ Istart / [0-9]
		
		Literal  $\leftarrow$ ['](!['] Char)* ['] Space
		
		\textbf{/} ["](!["] Char)* ["] Space
		
		Class $\leftarrow$ '[' (!'[' Range)* ']' Space
		
		Range $\leftarrow$ Char '-' Char 
		
		\textbf{/} Char
		
		Char $\leftarrow$ ' $\backslash\backslash$' [nrt'"$\backslash$ [ $\backslash$ ] $\backslash\backslash$ ]
		
		\textbf{/} '$\backslash\backslash$' [0-2][0-7][0-7] 
		
		\textbf{/}  '$\backslash\backslash$' [0-2][0-7]?
		
		\textbf{/}	!  '$\backslash\backslash$'

	\column{0.5\textwidth}
		\tiny
		Larrow $\leftarrow$ '<-' Space
		
		Slash $\leftarrow$ '/ ' Space
		
		And $\leftarrow$ '\&' Space
		
		Not $\leftarrow$ '!' Space
		
		Question $\leftarrow$ '?' Space
		
		Star $\leftarrow$ '*' Space
		
		Plus $\leftarrow$ '+' Space
		
		Open $\leftarrow$ '(' Space
		
		Close $\leftarrow$ ')' Space
		
		Dot $\leftarrow$ '.' Space
		
		Space $\leftarrow$ (Ws /  Comment)*
		
		Comment $\leftarrow$ '\#' (!EOL .)* EOL
		
		Space $\leftarrow$ '  ' \textbf{/}	'$\backslash$t' \textbf{/}
		
		EOL $\leftarrow$ '$\backslash$r$\backslash$n' \textbf{/} '$\backslash$n' \textbf{/} '$\backslash$r'
		
		EOF $\leftarrow$ ! .
		
		
			
\end{columns}
\end{minipage}
\end{frame}

\begin{frame}
	\frametitle{Unified language definition}
	
\begin{block}{}	The large majority of syntax description uses a context free grammar to specify the hierarchical structure and a set of regular expressions that specify the individual lexical elements.

\end{block}
\begin{block}{}
	Context Free Grammars cannot express all idioms, such as the greedy rule for identifiers and numbers, or "negative" syntax to described quoted string literals. 
\end{block}
\begin{block}{}
	Regular Expressions cannot describe recursive syntax.
\end{block}
\begin{block}{}
		Using both allows us to surpass the limitations.
\end{block}
\end{frame}
\begin{frame}
	\begin{block}{}
			These issues are non existent in PEGs, thanks to its operators. 
	\end{block}
	\begin{block}{}
		The greedy nature of the repetition operator ensures that a sequence of letters is always interpreted as a single identifier.  Negative syntax can be described thanks to the ! operators, as seen in the Char or Class definitions. 
	\end{block}
\begin{block}{}
	Lexical elements definition can refer to the hierarchical portion of the syntax
	\begin{center}
		\footnotesize
		Comment $\leftarrow$  ' (* '    ( Comment /  !  ’ *) ’   . )*        ’*)’
	\end{center}
\end{block}
\begin{block}{}
	This allows us to use a single, unified model to concisely express a machine-oriented language.
\end{block}
\end{frame}

\begin{frame}
	\frametitle{Handling ambiguity}

	\begin{block}{}
	 Ambiguity is usually resolved in CFGs by using informal meta-rules (e.g, for the dangling else problems, we assume that an if without else is prioritized)
\end{block} 
	 \begin{block}{}
	 	Thanks to prioritized choice, repetition operators and syntactic predicates, many ambiguities can  be entirely avoided in PEGs
	 \end{block}
 	\begin{center}
 		Statement $\leftarrow$ IF Cond THEN  Statement ELSE Statement 
 		
 		/	IF Cond THEN Statement  / 	$\dots$	
 	\end{center}
\end{frame}

\begin{frame}
	\frametitle{Limitations}
	\begin{block}{}
		Left recursion is still unavailable in PEGs. Because of the enforced priority,  a rule such as 
		\begin{center}A $\leftarrow$ A a / a\end{center}
		causes an infinite loop (can be avoided using repetition operators).
		
	\end{block}
	\begin{block}{}
		A parsing expression grammar is still a purely syntactic formalism. As such, it cannot fully express languages whose syntax depends on semantics predicates (e.g typedef identifiers in C) .  
	\end{block}
\end{frame}

\section{Formal analysis of PEGs}

\begin{frame}
	\begin{block}{}The definition of PEGs that we saw is very concrete and implementation oriented.\end{block}
	
	\begin{block}{}To reason about grammar and languages properties. we need to abstract way from the  unessential details and retain only the basic structure.\end{block}
\end{frame}

\begin{frame}
	\frametitle{Formal definition}
	\emph{Parsing Expression Grammar}
	
	\begin{center}
		G = $(V_n,V_t,R, e_S)$
	\end{center}

\begin{itemize}
	\item $V_n$ set of nonterminal symbols
	\item $V_t$ set of terminal symbols
	\item $R$ finite set of rules 
	\item $e_S$ the \emph{start expression}
\end{itemize}

$V_n \cap V_t = \emptyset$


\end{frame}
\begin{frame}
	$r \in R$ is a pair $(A,e)$, written $A \leftarrow e$, where $A \in V_n$ and $e$ is a parsing expression
	
	R is a function, meaning that for any nonterminal $A \in V_t$ there is one and only one $e$ such that $A \leftarrow e \in R$. We identify with $R(A)$ the  expression $e$ associated with A.
	
	R being functional prevents expression from containing undefined references or subroutine failures.
\end{frame}
\begin{frame}
	Parsing expression are defined inductively. 
	
	It is a parsing expression:
	\begin{itemize}
		\item The empty string $\epsilon$. 
		\item Any terminal $a \in V_t$.
		\item Any non terminal $A \in V_n$.
		\item The sequence $e1e2$, if $e_1$ and $e_2$ are parsing expression.
		\item The choice $e1/e2$, if $e_1$ and $e_2$ are parsing expression.
		\item The repetition $e^*$, if  e is a parsing expression.
		\item The not-predicate $!e$, if $e$ is a parsing expression.
	\end{itemize}
\begin{block}{}
	E(G):  a set  containing $e_S$, the expressions used in R and all their subexpressions.
\end{block}
\end{frame}

\begin{frame}
	\begin{block}{}
		\textbf{Repetition free grammar}: grammar where no expression contains the * operator.
		
	\end{block}

	\begin{block}{}
		\textbf{Predicate free grammar}: grammar where no expression contains the ! operator.
		
	\end{block}
\end{frame}

\begin{frame}
	\frametitle{Desugaring operators}
	The abstract syntax does not define  ' . ', ' + ', ' ? , ' \& ' and character classes.
	These constructs are desugared with the following rules:\begin{itemize}
		\item . $\rightarrow$ [a] $\forall a \in V_T$.
		\item $[a_1,a_2, ... , a_n] \rightarrow a_1 / a_2 / ... / a_n$.
		\item  	$e? \rightarrow e_d / \epsilon$  where $e_d$ is the desugaring of $e$.
		\item $ e^+ \rightarrow e_de_d^*$.
		\item $\&e \rightarrow !(!e)$.
\end{itemize} 
\end{frame}
\begin{frame}
	
	\frametitle{Grammar interpretation}
	We formalize the meaning of a grammar G defining the relation $\Rightarrow_G$.
	
	$\Rightarrow_G$ goes from pairs $(e,x)$ to pairs $(n,o)$, where $e$ is a parsing expression, $x \in V_t^*$ is an input string, $n \geq 0$ is a "step counter"  and $ o \in V_T^* \cup {f}$ is the result of a recognition  attempt (f represents a failure). 
	
	$\Rightarrow_G$ defined inductively.
\end{frame}
\begin{frame}
	\scriptsize
	\begin{columns}
		\column{0.5\textwidth}
		\begin{itemize}
		\item $(\epsilon, x) \Rightarrow  (1,\epsilon) \forall x \in V_t^*$.
		
		\item $(a,ax) \Rightarrow (1, a)$ if $A \in V_t,x \in V_t^*$.
		
		\item  $(a,bx) \Rightarrow (1, f) $if $ a \neq b$ and $(a, \epsilon) \Rightarrow (1,f)$.
		
		\item  $(A,x) \Rightarrow (n+1,o)$ if $A \Rightarrow e \in R$ and $(e,x) \Rightarrow (n,o)$.
		
		\item If $(e_1,x_1x_2y) \Rightarrow (n_1, x_1)$ and $(e_2,x_2y) \Rightarrow (n_2,x_2)$, then 
	$(e_1e_2,x_1x_2y) \Rightarrow (n_1+n_2+1,x_1x_2)$.
		
		\item If $(e_1, x) \Rightarrow (n_1, f)$, then $(e_1e_2,x) \Rightarrow (n_1+1,f)$
		
		\item If $(e_1,x_1y) \Rightarrow (n_1, x_1)$ and $(e_2, y) \Rightarrow (n_2, f)$, then $(e_1e_2, x_1y) \Rightarrow (n_1 + n_2 + 1, f)$
		\end{itemize}
	
	\column{0.5\textwidth}
	\begin{itemize}
			
		
		\item If $(e_1,xy) \Rightarrow (n_1,x)$, then $(e_1 / e_2, xy) \Rightarrow (n_1,x)$.
		
		\item If $(e_1,x) \Rightarrow (n_1,f)$ and $(e_2,x) \Rightarrow (n_2,o)$, then $(e_1 / e_2, x) \Rightarrow (n_1 + n_2 +1,o)$.
		
		\item If $(e,x_1x_2y) \Rightarrow (n_1,x_1)$ and $(e^*,x_2y) \Rightarrow (n_2,x_2)$, then $(e^*,x_1x_2y) \Rightarrow (n_1 + n_2 + 1, x_1x_2)$.
		
		\item  If $(e, x) \Rightarrow (n_1, f)$, then  $(e^*, x) \Rightarrow (n_1 + 1, \epsilon)$.
		
		\item If $(e, xy) \Rightarrow (n, x)$, then $(!e, xy) \Rightarrow (n+1, f)$
		
		\item If $(e, x) \Rightarrow (n, f)$, then $(!e, x) \Rightarrow (n+1, f)$.
	\end{itemize}
	\end{columns}
\end{frame}

\begin{frame}
	\small
	\frametitle{Languages and grammars properties}
	\begin{block}{}
		\emph{A language L is a parsing expression language (PEL) iff there exists a parsing expression grammar  G whose language is L}. 
	\end{block}
\begin{block}{Theorem}
	The class of parsing expression languages is close dunder intersection, union, and complement
\end{block}
\begin{block}{Proof}

Suppose that we  have $G_1 = (V_N^1,V_T, R^1,e_S^1)$, $G_2 = (V_N^2,V_T, R^2,e_S^2)$ that recognizes $L(G_1)$ and $L(G_2)$. Assuming that $V_N^1 \cap V_N^2 = \emptyset$, we define the grammar $G=(V_N^1 \cup V_N^2, V_T,R^1 \cup R_2, e_S')$ where $e_S'$ is one of the following:\begin{itemize}
	\item If $e_S' = e_S^1 / e_S^2$, then $L(G) = L(G_1) \cup L(G_2)$
	
	\item If $e_S' = \&e_S^1e_S^2$, then $L(G) = L(G_1) \cap L(G_2)$
	
	\item If $e_S' = !e_S^1$, then $L(G) = V_T^* - L(G_1)$
\end{itemize}
	
\end{block}
\end{frame}

\begin{frame}

	\small
	

	\begin{block}{}
	The class of parsing expression  includes non context-free languages. As an example, the language $a^nb^nc^n$ can be recognized using a PEG $G = ({A,B,D},{a,b,c},R,D)$ where R is composed by the following rules:
	\begin{align*}
		A  &\leftarrow  a A b / \epsilon \\
		B &\leftarrow bBc / \epsilon \\
		D &\leftarrow\&(A !b)a^*B!.
	\end{align*}
	\end{block}
		\begin{block}{}
		It is undecidable whether the language L(G) of an arbitrary grammar G is empty. Because of this, it is also undecidable the problem of knowing if a grammar G is complete.
	\end{block}
	\begin{block}{}
		We call a grammar \emph{well formed} if it does not have left recursive rules.
	\end{block}
\end{frame}


\begin{frame}

	
	$e_1(e_2e_3) \asymp (e_1e_2)e_3$\hfill $e_1(e_2 / e_3) \asymp (e_1 / e_2)e_3$
		
		
		
		$e_1(e_2 / e_3) \asymp (e_1/ e_2) / (e_1e_3)$  	\hfill $e_1(e_2 / e_3) \not\asymp (e_1/ e_3) / (e_2e_3)$
		
	

		
	$ e_1!e_2 \asymp   !(e_1e_2)e_1$ \hfill
	
	  $e_1$ and $e_2$ are disjoint if they succeed on disjoint sets of input strings.
	
	$e_1 / e_2 \asymp  e_2 / e_1$ if  $e_1$ and $e_2$ are disjoint.
	

	
\end{frame}
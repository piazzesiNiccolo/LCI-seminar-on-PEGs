\documentclass{beamer}
\usetheme{Frankfurt}
\usecolortheme{seahorse}
\title{Parsing Expression Grammars}
\author{Niccolò Piazzesi}
\institute[UniPi]{
    Università degli Studi di Pisa \\
    Anno Accademico 2020-21
}
\beamertemplatenavigationsymbolsempty
\setbeamertemplate{footline}[page number]
\AtBeginSection[]{
  \begin{frame}
  \vfill
  \centering
  \begin{beamercolorbox}[sep=8pt,center,shadow=true,rounded=true]{title}
    \usebeamerfont{title}\insertsectionhead\par%
  \end{beamercolorbox}
  \vfill
  \end{frame}
}

\begin{document}
    \begin{frame}
        \maketitle
    \end{frame}
    \section{Parsing}
    \begin{frame}
        \begin{block}{What's a parser?}
            Part of the compiler.
            Checks the stream of tokens produced by the \emph{lexer} for syntactical errors
            and produces an IR representation (usually an abstract syntax tree) of the source 
            that is used in later steps to generate machine code.
        \end{block}
        \begin{figure}
            \includegraphics[width=\textwidth]{img/parser.jpg}
        \end{figure}
    \end{frame}
    \begin{frame}
        \frametitle{Context free grammars}
        
        \begin{block}{}We need a model to describe a language syntax.\end{block}
        \pause
        \begin{block}{}Classical problem, solved by using formal language theory.\end{block}
        \pause
        \begin{block}{}{Syntax described by a \textbf{Context Free Grammar}.}\end{block}

    \end{frame}
    \begin{frame}
        \frametitle{CFG: definition}
        Context free grammar defined as:
        \begin{center}
            \textbf{G = (V,$\Sigma$, P, S)}
        \end{center}
        \begin{itemize}
            \item V = variables
            \item $\Sigma$ = terminals
            \item P = productions (or rules)
            \item S = start variables
        \end{itemize}
        Productions are in the form A $\rightarrow \alpha$, where A$\in$ V, $\alpha \in (V \cup \Sigma)$

        S $\Rightarrow$ A $|$ bb
        
        A $\Rightarrow$ B $|$ b 
       
        B $\Rightarrow$ S $|$ a
    \end{frame}
    \begin{frame}
        
            \begin{block}{}
            Given the rules of grammar G, we want to find a sequence of productions that 
            generates a target expression (\textbf{derivation}).
        
            
            \end{block}
            \begin{block}{}
            Parsing is the process of discovering such sequence.
            \end{block}
            
            $ S \rightarrow \gamma_{1} \rightarrow \gamma_{2} \rightarrow \dots \rightarrow \gamma_{n} \rightarrow expression$
            
            \begin{block}{}\textbf{Leftmost derivation}: at each step expand the leftmost non-terminal symbol\end{block}
            
            \begin{block}{}\textbf{Rightmost derivation}: at each step expand the rightmost non-terminal symbol\end{block}
            
    \end{frame}
    \begin{frame}
       \begin{columns}
           \begin{column}{0.4\textwidth}
               \begin{figure}
                   \includegraphics[width=\textwidth]{img/cfg.png}
                   \caption{classic grammar for arithmetic expressions}
               \end{figure}
           \end{column}
           \begin{column}{0.6\textwidth}
            \begin{figure}
                \includegraphics[width=\textwidth]{img/parse.png}
                \caption{parse tree for the expression x-2*y}
            \end{figure}
               
           \end{column}
       \end{columns}
    \end{frame}
    \section{Formal definiton of PEG}
    \section{A concrete example: \emph{pgen}}
\end{document}